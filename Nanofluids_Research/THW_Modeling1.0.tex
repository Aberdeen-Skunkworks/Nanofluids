\documentclass{article}
\usepackage{graphicx}
\usepackage{siunitx}
\usepackage{soul}
\usepackage{color}
%cover page
\author{Asad Mohiuddin}
\title{THW Modelling v1.0}


\begin{document}
\setlength{\parindent}{0pt}
%title page
\clearpage \maketitle
\thispagestyle{empty} %pervent page numbering

%start
\newpage \setcounter{page}{1}
\section{Ideal Analytical Model}
\subsection{Assumptions}
To start the simplest model we apply the following assumptions:
	\begin{itemize}
		\item{The thermo-physical properties of the medium are always constant.}
		\item{Hot wire has negligible heat capacity and a large thermal conductivity.}
		\item{No thermal resistance exists between the contact of the medium and surface of the hot wire.} 
		\item{Pure conductive heat transfer from an infinity long wire that has a  infinitely thin radius to an infinite surrounding medium.}
	\end{itemize}

Primarily there is heat transfer to the medium where the temperature of the hot wire varies with time. The heat transfer can be treated as a boundary condition as we are relative to the medium. The basic problem is driven by the non-stationary heat diffusion of Fourier's law

$$\rho C_p \frac{\partial T}{\partial t} = \lambda \nabla^2T $$
 \begin{itemize}
 	\item[] $\lambda$ is the thermal conductivity of the medium.
 	\item[] $\rho$ is the density of the medium.
 	\item[] $C_p$ is the specific heat capacity of the medium.
 \end{itemize}

Substituting in for the constants

$$ \frac{\partial T}{\partial t} = \alpha \nabla^2T $$

Where $\alpha = \frac{\lambda}{\rho C_p}$ the thermal diffusivity of the medium. The ability of the medium to conduct thermal energy, relative to its ability to store it.

\subsection{Boundary conditions}
\textbf{(1)} For any radial distance $r$ at $t \leq 0$ 
$$\Delta T_{ideal} (r, t) = 0$$
This boundary condition states that at the start the wire is in thermal equilibrium with the surrounding medium.

\vspace{3mm}

\textbf{(2)} At $r = 0 $ and $t \geq 0$
$$ \displaystyle \lim_{r\to \infty} \left(r \frac{\delta T}{\delta r} \right) = -\frac{q}{2 \pi \lambda}$$
\hl{Where $q$ is the power input per unit length of wire. This describes the radial temperature distribution as we get closer to the wire.}

\newpage

\textbf{(3)} At $r = \infty$ and $t \geq 0$ 
$$\displaystyle \lim_{r\to \infty} \Delta T_{ideal}(r,t) = 0$$
\hl{A long distance away from the wire the temperature profile remains constant.}

\subsection{Solution to the Fourier equation}

The solution to this problem is well known and was analytically proven by Carlslaw H S and Jaeger J C in 1959, they found

$$\Delta T_{ideal}(r,t) =  \frac{q}{4 \pi \lambda}  E_1 \left( \frac{4 \alpha t}{r^2}\right)$$

The expression $E_1 \left( \frac{4 \alpha t}{r^2}\right)$ is the exponential integral and the series expansion of this exponential integral is 

$$\Delta T_{ideal}(r,t) = \frac{q}{4 \pi \lambda} \left[ \ln \left( \frac{4 \alpha t}{r^2 C}\right) + \frac{\left( \frac{r^2}{4 \alpha t} \right)}{1(1!)} + \frac{\left( \frac{r^2}{4 \alpha t} \right)}{2(2!)} + ... \right]$$

Where $ C = e^{\gamma} = 1.781 $ is the exponential of Euler's constant. 

\vspace{3mm}

For sufficiently long times (i.e. small values of $\frac{r^2}{4 \alpha t}$) where $t \gg \frac{r^2}{4 \alpha}$ (quantitatively where $t \geq 10ms$). \hl{Does this mean that for very short times (short enough to see thermophersis phenomenon) we should include the series expansion?} Also that the surface of the hot wire imposes a uniform temperature which is equal to that in the conducting medium at $r = r_0$. The ideal equation describing the temperature history of the wire 

\begin{equation} \label{eq:ideal}
\Delta T_{ideal}(r_0,t) =  \frac{q}{4 \pi \lambda} \ln \left( \frac{4 \alpha t}{r^2_0 C}\right)
\end{equation}

That can also be expressed as:
	\begin{equation} \label{eq:ideal2}
	\Delta T_{ideal}(r_0,t) =  \frac{q}{4 \pi \lambda} \ln \left( \frac{4 \alpha }		{r^2_0 C}\right) + \frac{q}{4 \pi \lambda} \ln \left( t \right)
	\end{equation}

At a fixed radial distance within two instances of time, the temperature change of the wire

$$\Delta T_{2,ideal} - \Delta T_{1,ideal} = \frac{q}{4 \pi \lambda} \ln \left( \frac{t_2}{t_1} \right)$$

Where subscripts 1 and 2 represent the first and second experimental reading respectively. The ideal thermal conductivity of the medium can be found as

$$\lambda_{ideal} =  \frac{q}{4 \pi \left( \Delta T_{2,ideal} - \Delta T_{1,ideal} \right)}\ln \left( \frac{t_2}{t_1} \right)$$

\newpage

	\begin{equation} \label{eq:idealconductivity}
\lambda_{ideal} =  \frac{q}{4 \pi} \frac{d\ln \left( t \right)}{d\Delta T_{ideal}}
	\end{equation}
	
At a fixed radial distance the thermal conductivity is defined as:
$$\lambda_{ideal} =  \frac{q}{4 \pi} \frac{d\ln \left( t \right)}{d\Delta T_{ideal}}$$
The typical ideal plot for the temperature change of the wire against the logarithmic time, for an ideal experiment of the THW will propose a straight line.

\subsection{Technique for determining ideal thermal conductivity}
\subsubsection{THW: For a single hot wire}
Because of the well established temperature-resistance relationship of the wire, not only can it be used as a source of heat into the bulk medium but also as a temperature sensor.

The wire’s temperature will change with time as there is a steady flow of heat into the  surrounding medium. As a result the electrical resistance, R, of the wire changes.
\hl{Resistance of the wire can then be expressed as}

\hl{[http://hyperphysics.phy-astr.gsu.edu/hbase/electric/restmp.html]}
	
	\begin{equation} \label{Resistance}
	R(t) = R_0 \left[ 1 + \sigma \Delta T_{ideal}(t) \right]
	\end{equation}


 	\begin{itemize}
 		\item[] \hl{$R_0$ is the initial resistance at room temperature.}
 		\item[] $\sigma$ is the temperature coefficient of resistance of the wire. How do we find a value for this? \hl{See Nist paper for in situ calibration of the wire.}
 	\end{itemize}

From Ohms law, for a fixed current $I$, the potential difference across the terminal of the wire will be
	\begin{equation} \label{eq:PD}
	V(t) = R(t)I = R_0 \left[ 1 + \sigma \Delta T_{ideal}(t) \right] I
	\end{equation}

Substituting equation \ref{eq:ideal2} into equation \ref{eq:PD} yields
	\begin{equation} \label{eq:PD2}
	V(t) = R_0 \left[ 1 + \sigma \frac{q}{4 \pi \lambda} \ln \left( \frac{4 \alpha }		{r^2_0 C}\right) + \sigma \frac{q}{4 \pi \lambda} \ln \left( t \right) \right] I
	\end{equation}

The heat produced per unit length in the wire $q$, comes from the loss of potential energy of the flow of electrons $I$ as they move from the positive terminal of the wire to the negative terminal. By the Joule's effect the energy is converted to thermal energy from the collisions of electrons within the crystal lattice.  
For an ideal case we can say that the energy per unit length $L$ produced as Joules heat into the medium for a fixed current  can be expressed as

$$q = \frac{IV_0}{L}$$

Through Ohms law 

	\begin{equation} \label{eq:idealheat}
	q = \frac{I^2R_0}{L}
	\end{equation}
	
\newpage

Substituting equation \ref{eq:idealheat} into equation \ref{eq:PD2}

$$V(t) = R_0 \left[ 1 + \sigma \frac{I^2R_0/L}{4 \pi \lambda} \ln \left( \frac{4 \alpha } {r^2_0 C}\right) + \sigma \frac{I^2R_0/L}{4 \pi \lambda} \ln \left( t \right) \right] I$$
	
	\begin{equation} \label{eq:PD3}
V(t) = R_0 I + \sigma \frac{I^3R_0^2/L}{4 \pi \lambda} \ln \left( \frac{4 \alpha } {r^2_0 C}\right) + \sigma \frac{I^3R_0^2/L}{4 \pi \lambda} \ln \left( t \right)
	\end{equation}

Replacing for the constants 
	\begin{equation} \label{eq:PD3}
V(t) = A + B \ln \left( t \right)
	\end{equation}

Where $ A =  R_0 I + \sigma \frac{I^3R_0^2/L}{4 \pi \lambda} \ln \left( \frac{4 \alpha } {r^2_0 C}\right) $ and $ B = \sigma \frac{I^3R_0^2/L}{4 \pi \lambda}$

Equation \ref{eq:PD3} shows that for an ideal experiment the voltage should be linear with respect to the logarithmic time.

Differentiating \ref{eq:PD3} with respect to the log of time
$$\frac{dV(t)}{d \ln (t)} = B = \frac{\sigma I^3R_0^2}{4 \pi \lambda L}$$

Solving for the ideal thermal conductivity

$$ \lambda_{ideal}= \frac{4 \pi L}{\sigma I^3R_0^2} \frac{dV(t)}{d \ln (t)} $$

Where $\frac{dV(t)}{d \ln (t)} $ is the gradient of the line and is expressed as a constant.

\subsubsection{THW: For a Wheatstone bridge}

	\begin{figure}[h]
		\centering
		\includegraphics[scale=0.7]{WBPIC3}
		\caption{\textit{THW Wheatstone bridge schematic}}
		\label{fig:pcbdesign}
	\end{figure}

The Wheatstone bridge is a classical technique that is commonly applied for temperature measurements. A major challenge to achieve accurate resistance measurements is to nullify the loading effect of the circuit from the digital voltmeter (DVM). These inaccuracies arise since the DVM draws a small amount of power from the circuit, even though the DVM has a relatively high impedance (e.g 50$M\Omega$). When the Wheatstone bridge is balanced the DVM does not draw any power from the circuit hence increasing accuracy by nullifying the loading effect. 

\vspace{3mm}

The relationship that describes the potential difference across Node A and Node B with respect to time  

	\begin{equation} \label{V_wb}
V_{AB}(t) = \left( \frac{R_1 + R_l(t)}{R_1 + R_l(t) + R_3} - \frac{R_4}{R_2 + R_s(t) + R_4} \right) V_{s}
	\end{equation}

Recalling equation \ref{Resistance} and \ref{eq:ideal2} we can write the the resistance of the long hot wire $R_l(t)$ as

$$R_l(t) = R_{l,0} \left[ 1 + \sigma \frac{q_l}{4 \pi \lambda} \ln \left( \frac{4 \alpha }		{r^2_0 C}\right) + \sigma \frac{q_l}{4 \pi \lambda} \ln \left( t \right) \right]$$

	\begin{equation} \label{eq:R_long}
	R_l(t) = A_l + B_l \ln (t)
	\end{equation}
Where $A_l = R_{1,0} + R_{1,0} \sigma \frac{q_l}{4 \pi \lambda} \ln \left( \frac{4 \alpha }		{r^2_0 C}\right)$ and $B_l = R_{1,0} \sigma \frac{q_l}{4 \pi \lambda}$


\vspace{3mm}

And for the short hot wire $R_s(t)$

$$R_s(t) = R_{s,0} \left[ 1 + \sigma \frac{q_s}{4 \pi \lambda} \ln \left( \frac{4 \alpha }		{r^2_0 C}\right) + \sigma \frac{q_s}{4 \pi \lambda} \ln \left( t \right) \right]$$
	
	\begin{equation} \label{eq:R_short}
	R_s(t) = A_s + B_s \ln (t)
	\end{equation}

Where $A_s = R_{s,0} + R_{s,0} \sigma \frac{q_s}{4 \pi \lambda} \ln \left( \frac{4 \alpha }		{r^2_0 C}\right)$ and $B_s = R_{s,0} \sigma \frac{q_s}{4 \pi \lambda}$

\vspace{3mm}

Substituting equation \ref{eq:R_long} and \ref{eq:R_short} into equation \ref{V_wb}
$$V_{AB}(t) = \left( \frac{R_1 + A_l + B_l \ln (t)}{R_1 + A_l + B_l \ln (t) + R_3} - \frac{R_4}{R_2 + A_s + B_s \ln (t) + R_4} \right) V_{s}$$

Replacing constants 



\newpage

\section{Experimental corrections }
The THW (including the cell) design is to mimic the ideal behaviour as much as possible. However, realistically it is impossible to create the perfect design. Therefore we need to make corrections that account for the deviation between the real instrument and the ideal model.

To achieve the ideal temperature rise we have to add a number of corrections to the experimentally determined temperature rise

$$ \Delta T_{ideal} = \Delta T_{experimental} + \sum_{i} \delta T_i$$

\subsection{Finite radius of the wire}
\subsection{Composite cylinders}
\subsection{Knudsen effect}
\subsection{Radiation}
\subsection{Time dependant heat flux}
The ideal analysis imposed that the applied heat flux $q$ of the wire remained constant during the measurement. Depending on the power source, thermal expansion of the wire, and the magnitude of the overall resistance changes during the run this may not be an adequate assumption.

$$q(t) = \frac{I^2 R}{L}$$

Provided that we have a platinum wire we can neglect the effect of thermal expansion as it its thermal expansion coefficient is small. The instantaneous heat flux can be represented as [Taken from NIST]

$$ q(t) = \left( \frac{V_s(t)}{R_1 + R_2} \right) ^2 \frac{R_l(t) + R_s(t)}{L_l + L_s}$$

where $V(t)$ is the voltage applied to the bridge, and $L$ is the corresponding length of the wire.
A correction can be applied to $\Delta T_{experimental}(t)$ by the ratio of instantaneous $q(t_i)$ to the heat flux at the median time of the experiment $q(t_a)$.

$$\Delta T_{corr}(t) = \Delta T_{experimental}(t) * \left( \frac{q(t_i)}{q(t_a)} \right)$$

Where the median time is expressed as $t_a = (t_1 + t_2)/2$.
At room temperature the drop in current is almost eactly offset by the rise in the hot wire resistance. But at lower temperatures the change in hot wire resistance predominates [NIST].
\end{document}